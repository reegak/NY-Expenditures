% Stat 696: Knitr lab
% Illustrating knitr to present analyses of College data from ISLR text
% Packages required: knitr, xtable, stargazer, ISLR
% To use, make sure to call library(knitr) first in console
% To run and create a .tex file: knit('knitr_ClassVersion.Rnw') in R
% August 25, 2017


% To show at start of class:
% -- Step through preface briefly and show where to enter name
% -- LaTeX preface (preface.tex) vs. Knitr preface (knitr_ClassVersion.Rnw);
%    briefly delineate two approaches to report writing
% -- Code in place for regression analysis and prediction
% -- There are 6 tasks to complete with sample code and hints where needed
% -- Suggest cutting an pasting R code into console first to debug


% Preface required in the knitr RnW file
\documentclass{article}\usepackage[]{graphicx}\usepackage[]{color}
%% maxwidth is the original width if it is less than linewidth
%% otherwise use linewidth (to make sure the graphics do not exceed the margin)
\makeatletter
\def\maxwidth{ %
  \ifdim\Gin@nat@width>\linewidth
    \linewidth
  \else
    \Gin@nat@width
  \fi
}
\makeatother

\definecolor{fgcolor}{rgb}{0.345, 0.345, 0.345}
\newcommand{\hlnum}[1]{\textcolor[rgb]{0.686,0.059,0.569}{#1}}%
\newcommand{\hlstr}[1]{\textcolor[rgb]{0.192,0.494,0.8}{#1}}%
\newcommand{\hlcom}[1]{\textcolor[rgb]{0.678,0.584,0.686}{\textit{#1}}}%
\newcommand{\hlopt}[1]{\textcolor[rgb]{0,0,0}{#1}}%
\newcommand{\hlstd}[1]{\textcolor[rgb]{0.345,0.345,0.345}{#1}}%
\newcommand{\hlkwa}[1]{\textcolor[rgb]{0.161,0.373,0.58}{\textbf{#1}}}%
\newcommand{\hlkwb}[1]{\textcolor[rgb]{0.69,0.353,0.396}{#1}}%
\newcommand{\hlkwc}[1]{\textcolor[rgb]{0.333,0.667,0.333}{#1}}%
\newcommand{\hlkwd}[1]{\textcolor[rgb]{0.737,0.353,0.396}{\textbf{#1}}}%
\let\hlipl\hlkwb

\usepackage{framed}
\makeatletter
\newenvironment{kframe}{%
 \def\at@end@of@kframe{}%
 \ifinner\ifhmode%
  \def\at@end@of@kframe{\end{minipage}}%
  \begin{minipage}{\columnwidth}%
 \fi\fi%
 \def\FrameCommand##1{\hskip\@totalleftmargin \hskip-\fboxsep
 \colorbox{shadecolor}{##1}\hskip-\fboxsep
     % There is no \\@totalrightmargin, so:
     \hskip-\linewidth \hskip-\@totalleftmargin \hskip\columnwidth}%
 \MakeFramed {\advance\hsize-\width
   \@totalleftmargin\z@ \linewidth\hsize
   \@setminipage}}%
 {\par\unskip\endMakeFramed%
 \at@end@of@kframe}
\makeatother

\definecolor{shadecolor}{rgb}{.97, .97, .97}
\definecolor{messagecolor}{rgb}{0, 0, 0}
\definecolor{warningcolor}{rgb}{1, 0, 1}
\definecolor{errorcolor}{rgb}{1, 0, 0}
\newenvironment{knitrout}{}{} % an empty environment to be redefined in TeX

\usepackage{alltt}

\usepackage{rotating}
\usepackage{graphics}
\usepackage{latexsym}
\usepackage{color}
\usepackage{listings} % allows for importing code scripts into the tex file

% Approximately 1 inch borders all around
\setlength\topmargin{-.56in}
\setlength\evensidemargin{0in}
\setlength\oddsidemargin{0in}
\setlength\textwidth{6.49in}
\setlength\textheight{8.6in}

% Options for code listing; from Patrick DeJesus, October 2016
\definecolor{codegreen}{rgb}{0,0.6,0}
\definecolor{codegray}{rgb}{0.5,0.5,0.5}
\definecolor{codepurple}{rgb}{0.58,0,0.82}
\definecolor{backcolour}{rgb}{0.95,0.95,0.92}
\lstdefinestyle{mystyle}{
	backgroundcolor=\color{backcolour},   commentstyle=\color{codegreen},
	keywordstyle=\color{magenta},
	numberstyle=\tiny\color{codegray},
	stringstyle=\color{codepurple},
	basicstyle=\footnotesize,
	breakatwhitespace=false,         
	breaklines=true,                 
	captionpos=b,                    
	keepspaces=true,                 
	numbers=left,                    
	numbersep=5pt,                  
	showspaces=false,                
	showstringspaces=false,
	showtabs=false,                  
	tabsize=2
}
%"mystyle" code listing set
\lstset{style=mystyle}
%\lstset{inputpath=appendix/}


\title{Stat 696, Example Application of \texttt{knitr}} 
\author{Kelso Quan}
\date{\today}
\IfFileExists{upquote.sty}{\usepackage{upquote}}{}
\begin{document} 
\maketitle

% Code to start knitr


%%%%%%%%%%%%%%%%%%%%%%%%%%%%%%%%%%%%%%%%%%%%%%%%%%%%%%%%%%%%
% Code snippet to load in libraries and data  
% THIS IS HOW R-CODE IS READ INTO LaTeX DOC WITH knitr
% Environment:  
% <<...>>=  
% [Code here] 
% @
%%%%%%%%%%%%%%%%%%%%%%%%%%%%%%%%%%%%%%%%%%%%%%%%%%%%%%%%%%%%


  
% Code snippet to run the regression analysis, including prediction for two new Universities



%%%%%%%%%%%%%%
% Lab Tasks
%%%%%%%%%%%%%%

%% Task 1: Present an R dump of the summary of the regression fit
% [Place knitr code chunk here]
\begin{knitrout}
\definecolor{shadecolor}{rgb}{0.969, 0.969, 0.969}\color{fgcolor}\begin{kframe}
\begin{alltt}
\hlkwd{summary}\hlstd{(fm1)}
\end{alltt}
\begin{verbatim}
## 
## Call:
## lm(formula = Apps ~ Private + Elite + Accept + Outstate + Room.Board + 
##     Grad.Rate, data = College)
## 
## Residuals:
##     Min      1Q  Median      3Q     Max 
## -5094.5  -329.7   -22.6   226.8 10114.6 
## 
## Coefficients:
##               Estimate Std. Error t value Pr(>|t|)    
## (Intercept) -985.95379  204.82380  -4.814 1.78e-06 ***
## PrivateYes  -291.62328  133.28335  -2.188 0.028970 *  
## EliteYes    1745.00184  151.74052  11.500  < 2e-16 ***
## Accept         1.42869    0.02024  70.601  < 2e-16 ***
## Outstate      -0.01427    0.01690  -0.845 0.398600    
## Room.Board     0.16615    0.04953   3.355 0.000834 ***
## Grad.Rate      8.63483    2.94718   2.930 0.003491 ** 
## ---
## Signif. codes:  0 '***' 0.001 '**' 0.01 '*' 0.05 '.' 0.1 ' ' 1
## 
## Residual standard error: 1128 on 770 degrees of freedom
## Multiple R-squared:  0.9157,	Adjusted R-squared:  0.915 
## F-statistic:  1394 on 6 and 770 DF,  p-value: < 2.2e-16
\end{verbatim}
\end{kframe}
\end{knitrout}
% ... allows us to run R-code or grab R elements inside the text.
% Try it out!  Write a sentence using \Sexpr to grab the predicted values for 
% the new schools (variables newpred1 and newpred2 from above).
The new predicted values: newpred1: 7000, 4800, 9300 and newpred2: 2300, 57, 4600 from the data.

%% Task 2: Insert a pairwise scatterplot into your document
% For plots, start by setting up the LaTeX figure environment,
% then place R code to knit, then set up LaTeX code to complete figure environment.
% Below I give the code for this task.  You will practice with this code in Task 5.
\begin{figure}
\begin{center}
\begin{knitrout}
\definecolor{shadecolor}{rgb}{0.969, 0.969, 0.969}\color{fgcolor}
\includegraphics[width=4in]{figure/unnamed-chunk-2-1} 

\end{knitrout}
\caption{Matrix of variables}
\label{matrix}
\end{center}
\end{figure}

% Write a short blurb of text to cite your figure.
There is a comparison of every variable correlation to others in figure \ref{matrix}.

%% Task 3: Use stargazer to present summary statistics of the College data set

% Table created by stargazer v.5.2.2 by Marek Hlavac, Harvard University. E-mail: hlavac at fas.harvard.edu
% Date and time: Sun, Sep 23, 2018 - 9:19:54 PM
% Requires LaTeX packages: rotating 
\begin{sidewaystable}[!htbp] \centering 
  \caption{Summary statistics for the ISLR College data set.} 
  \label{descrips} 
\begin{tabular}{@{\extracolsep{5pt}}lcccccc} 
\\[-1.8ex]\hline 
\hline \\[-1.8ex] 
Statistic & \multicolumn{1}{c}{Mean} & \multicolumn{1}{c}{St. Dev.} & \multicolumn{1}{c}{Min} & \multicolumn{1}{c}{Pctl(25)} & \multicolumn{1}{c}{Pctl(75)} & \multicolumn{1}{c}{Max} \\ 
\hline \\[-1.8ex] 
Applicants & 3,001.638 & 3,870.201 & 81 & 776 & 3,624 & 48,094 \\ 
TotalAccepted & 2,018.804 & 2,451.114 & 72 & 604 & 2,424 & 26,330 \\ 
TotalEnrolled & 779.973 & 929.176 & 35 & 242 & 902 & 6,392 \\ 
Top10%HSstudents & 27.559 & 17.640 & 1 & 15 & 35 & 96 \\ 
Top25%HSstudents & 55.797 & 19.805 & 9 & 41 & 69 & 100 \\ 
FullTimeStudent & 3,699.907 & 4,850.421 & 139 & 992 & 4,005 & 31,643 \\ 
PartTimeStudent & 855.299 & 1,522.432 & 1 & 95 & 967 & 21,836 \\ 
OutofStateTution & 10,440.670 & 4,023.016 & 2,340 & 7,320 & 12,925 & 21,700 \\ 
RoomBoardPrices & 4,357.526 & 1,096.696 & 1,780 & 3,597 & 5,050 & 8,124 \\ 
BookPrices & 549.381 & 165.105 & 96 & 470 & 600 & 2,340 \\ 
PersonalExpense & 1,340.642 & 677.071 & 250 & 850 & 1,700 & 6,800 \\ 
%Facw/PhD & 72.660 & 16.328 & 8 & 62 & 85 & 103 \\ 
%Facw/TerminalDeg & 79.703 & 14.722 & 24 & 71 & 92 & 100 \\ 
S/F Ratio & 14.090 & 3.958 & 2.500 & 11.500 & 16.500 & 39.800 \\ 
%AlumniDonate & 22.744 & 12.392 & 0 & 13 & 31 & 64 \\ 
budgetPerStudent & 9,660.171 & 5,221.768 & 3,186 & 6,751 & 10,830 & 56,233 \\ 
Grad.Rate & 65.463 & 17.178 & 10 & 53 & 78 & 118 \\ 
\hline \\[-1.8ex] 
\end{tabular} 
\end{sidewaystable} 



%% Task 4: Create a table of predictions using xtable
% I provide the code below for a base table.  
% The task is then to add additional columns to the table and create the LaTeX code using xtable.
% Note that we use results="asis" to force knitr to present the table code for compiling in LaTeX
% latex table generated in R 3.5.1 by xtable 1.8-3 package
% Sun Sep 23 21:19:54 2018
\begin{table}[ht]
\centering
\begin{tabular}{|l|rrrr|}
  \hline
 & Estimate & Std. Error & t value & Pr($>$$|$t$|$) \\ 
  \hline
(Intercept) & -985.95 & 204.82 & -4.81 & 0.00 \\ 
  PrivateYes & -291.62 & 133.28 & -2.19 & 0.03 \\ 
  EliteYes & 1745.00 & 151.74 & 11.50 & 0.00 \\ 
  Accept & 1.43 & 0.02 & 70.60 & 0.00 \\ 
  Outstate & -0.01 & 0.02 & -0.84 & 0.40 \\ 
  Room.Board & 0.17 & 0.05 & 3.35 & 0.00 \\ 
  Grad.Rate & 8.63 & 2.95 & 2.93 & 0.00 \\ 
   \hline
\end{tabular}
\caption{Inferences from regressing number of applications on whether the college is private or public, whether the college is elite or not, acceptance rate, out of state tuition, room and board, and graduation rate.} 
\label{reginf}
\end{table}


% Write a short blurb citing your table.


%% Task 5: Create an appendix of plots
% We will create a 2x2 graphic of regression diagnostics
\newpage
\noindent \Large{{\bf Appendix A: Supplementary Plots}}
\begin{figure}[h!]
\begin{center}
%%%%%%%%%%%%%%%%
% Here is code for the default regression diagnostics from R
% Write knitr code to present a 2x2 graphic for this appendix.
% Suggestion: use the knitr code environment from the scatterplot matrix of Task 2
%
%  par(mfrow=c(2,2))
%  plot(fm1)
%
%%%%%%%%%%%%%%%%%%%
\begin{knitrout}
\definecolor{shadecolor}{rgb}{0.969, 0.969, 0.969}\color{fgcolor}\begin{kframe}
\begin{alltt}
  \hlkwd{par}\hlstd{(}\hlkwc{mfrow}\hlstd{=}\hlkwd{c}\hlstd{(}\hlnum{2}\hlstd{,}\hlnum{2}\hlstd{))}
  \hlkwd{plot}\hlstd{(fm1)}
\end{alltt}
\end{kframe}
\includegraphics[width=\maxwidth]{figure/graphics-1} 

\end{knitrout}
\caption{Supplemental figures}
\label{figures}
\end{center}
\end{figure}

% Write a short blurb citing the figure and stating what it is.
Figure \ref{figures} shows us the residuals.

%% Task 6: Create an appendix of code
% Here is the LaTeX code from the online video.
% Recall that this is straight LaTeX, no knitr code chunk needed!
  \newpage
  \noindent \Large{{\bf Appendix B: R Code}}
  \lstinputlisting[language=R, caption = Knitr Lab]{knitr_lab.Rnw}


\end{document}
